\documentclass[slides]{pgnotes}

\title{Modelling}

\begin{document}

\maketitle

\tableofcontents

\section{E-R models}
\label{e-r-models}

An Entity-Relationship Model represents the data requirements for a
system.

\begin{redbox}{E-R models should help}
\begin{itemize}
\item It is important to note that ER models are a tool, not a rule.
\item Just like language, the same system may be adequately described by
multiple differing ER models which are all equally valid.
\end{itemize}
\end{redbox}

\subsection{Diagrammatic forms}
\label{diagrammatic-forms}

An Entity-Relationship Diagram is a \textbf{visual presentation} of the
entity-relationship model for a system.

\begin{greenbox}{E-R diagram types}

There are several different styles of diagram used:

\begin{description}
\item[Crow's foot]
  named due to similarity of the ``many'' symbol to a bird's foot.
\item[Unified Modelling Language (UML)]
  used in non-database
\end{description}

\end{greenbox}

The one we will study is the ``crow's foot'' notation.


\subsection{Synthesis}
\label{synthesis}

Synthesis is the transformation of an ER model to a relational (or
other) database or a combination of persistence technologies.

Some specific issues need to be dealt with the context of synthesising ER
diagrams to relational databases.


\section{Entities}
\label{entities}

Key terms:

\begin{description}
\item[Entity type:]
A group of objects with the same properties, wwhich are identified by
the enterprise as having an independent existence.
\item[Entity occurrence:]
A uniquely identifiable object of an entity type.
\end{description}

Often use ``Entity'' for both type and occurrence, although ambiguous,
where meaning is obvious.

Entities are represented by boxes in ER diagrams.

\newpage 

\begin{figure}[htbp]
\centering
\includegraphics{entities}
\caption{Entities as boxes}
\end{figure}

\section{Attributes}
\label{attributes}

\begin{description}
\item[Attribute type:]
a property of an entity type or relationship type.
\item[Attribute domain:]
set of allowable values for attribute.
\end{description}


\subsection{Simple / complex attributes}
\label{simple-complex-attributes}

\begin{description}
\item[Simple attribute]
composed of single component with independent existence (e.g postcode)
\item[Composite attribute]
composed of multiple components, each with independent existence.
(e.g.~address)
\end{description}

\subsection{Single / multi-valued attributes}
\label{single-multi-valued-attributes}



\subsection{Derived attributes}
\label{derived-attributes}

Derived attributes are those appearing as an attribute that are dynamically constructed from others:

For example a fullname constructed from firstname and lastname.


\section{Relationships}
\label{relationships}

Entities are linked together by relationships.

A relationship is an association between two entities that is important to the system of interest. 

\begin{description}
\item[Relationship type]
A set of meaningful associations among entity types
\item[Relationship occurrence]
A uniquely identifiable association that includes one occurrence from each participating entity type.
\end{description}

Often use ``Relationship'' for both type and occurrence, although
ambiguous, where meaning is obvious.

\subsection{Representation of a relationship}
\label{representation-of-a-relationship}

Relationships between entities are represented in an E-R diagram by a
line connecting both entities.

It can sometimes be helpful to write the verb beside the line, but it's not required.

Also if the verb makes sense in one direction, an arrow is drawn from subject to object.

\newpage 

\begin{figure}[htbp]
\centering
\includegraphics{relationship}
\caption{Relationship}
\end{figure}

\subsection{Degree of Relationship}
\label{degree-of-relationship}

The degree of a relationship type is the number of participating entity
types in a relationship {[}@connolly:2015:database{]}.

\begin{description}
\item[Binary relationship]
has exactly 2 participating entities.
\item[Complex relationship]
has degree greater than 2.
\end{description}

\subsection{Cardinality}
\label{cardinality}

The cardinality of a relationship specifies how many related entity
occurrences that an entity occurrence maps to, such as:

\begin{table}[htbp]
  \centering
  \begin{tabular}{l l}
    \toprule
    \textbf{Abbreviation} & \textbf{Meaning}\\
    \midrule
    1:m & One-to-Many\\
    m:1 & Many-to-one\\
    1:1 & One-to-one\\
    m:n & Many-to-many\\
    \bottomrule
  \end{tabular}
  \caption{Cardinalities}
  \label{tab:cardinalities}
\end{table}

\newpage

Symbols for drawing cardinalities are shown in :

\begin{figure}[htbp]
\centering
%\includegraphics{cardinality}
\caption{Representation of cardinality in Crow's Foot notation{}}
\end{figure}

If a NULL is allowed on one side of the relationship, then the minimum
cardinality is zero on the other side. The relationship becomes
zero-or-one-to-many, or zero-or-many-to-one etc.

\newpage

\begin{figure}[htbp]
\centering
%\includegraphics{cardinality_zero}
%\caption{Cardinality zero}
\end{figure}

\subsection{Synthesising m:n relationships}
\label{synthesising-mn-relationships}

Many-to-many relationships can't be directly implemented on a relational
database. Instead, they have to be decomposed into two simpler
relationships and a new entity has to be created:

\begin{enumerate}
\def\labelenumi{\arabic{enumi}.}
\item
  Introduce a new linking entity.
\item
  Form two 1:m relationships.
\end{enumerate}

\subsection{Implementation}
\label{implementation}

Roughly speaking:

\begin{itemize}
\item
  Entities become tables
\item
  Attributes become columns
\item
  1-to-many / Many-to-1 Relationships are represented as foreign keys.
  The foreign key is set on the ``many'' table pointing to the ``1''
  table.
\end{itemize}

\subsection{Relationship participation}
\label{relationship-participation}

Participation determines ``whether all or only some entity occurrences
particiate in a relationship''.

In this respect a relationship can be optional, mandatory or contingent.

\subsubsection{Optional}
\label{optional}

An optional relationship is one where the minimum cardinality on both
sides is zero.

Consider a car dealership:

\begin{itemize}
\item
  A customer may exist without a car. There may be customers on the
  books who have not purchased, but may purchase at a later date.
\item
  A car may exist without a customer. There are cars that have not been
  purchased yet.
\end{itemize}

\subsubsection{Mandatory}
\label{mandatory}

A mandatory relationship is one where the minimum cardinality on both sides is 1.

For example, a doctor must have patients and patients must have a doctor.

\subsubsection{Contingent}
\label{contingent}

A contingent relationship is one where:

\begin{enumerate}
\def\labelenumi{\arabic{enumi}.}
\item
  On one side, the minimum cardinality is zero.
\item
  On the other side, the minimum cardinality is one.
\end{enumerate}

Consider a trading business: A sales order must be associated with a
customer, but at any point in time there may be no outstanding sales
orders for some customers.

\subsection{Exclusive relationship}
\label{exclusive-relationship}

An exclusive relationship is said to exist when the existence of one
relationship precludes the existence of another.

For example, a patient may not be a patient in a ward and a patient in
the outpatient department.



\subsection{Recursive relationship}\label{recursive-relationship}

Relationship where same entity type participates more than once.


\end{document}



