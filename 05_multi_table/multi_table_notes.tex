\documentclass[slides]{pgnotes}

\title{Multi-table Databases}

\begin{document}

\maketitle

\tableofcontents


\section{Scenario}

A group of local artisan producers operate a popup stall for a 3-month period.
They contract a local IT expert to build a simple sales tracking and reporting system.

\begin{bluebox}{Main requirements}
  \begin{enumerate}
  \item Sales must be numbered and timestamped.
  \item Every sale is separate (no concept of a ``basket'' for simplicity!)
  \item Stall accepts cash and card tender.
    \begin{itemize}
    \item Card payments have a 1.68\% processor fee (which may vary).
    \item May be requirement to accept additional payment types during period.
    \end{itemize}
  \item Sale must capture producer, description, quantity sold, amount paid and tender.
  \item Everyone will be paid at the end of the 3-month period.
  \item Database must be able to produce the following reports:
  \end{enumerate}
\end{bluebox}


\subsection{Implementation with 3 tables}

\begin{center}
\includegraphics[width=0.6\linewidth]{scenario}
\end{center}

\subsection{Key decisions}

\begin{enumerate}
\item \textbf{Auto-numbered} \texttt{id} column using \texttt{SERIAL}.
\item \textbf{Primary key} will be the auto-numbered \texttt{id} column.
\item All fields will be \textbf{not null}.
\item Every sale must match a valid:
  \begin{description}
  \item[producer] from the \texttt{producer} table.
  \item[tender] from the \texttt{tender} table.
  \end{description}
\item Storage of numerical values:
  \begin{description}
  \item[Amounts] as \texttt{NUMERIC(6,2)} typed \texttt{amount}.
  \item[Rate] as \texttt{NUMERIC(6,2)} typed \texttt{rate}. 
  \end{description}
\end{enumerate}

\section{Domains}\label{sec:domains}

\href{https://www.postgresql.org/docs/13/domains.html}{Domains} are
user-defined types based on underlying type. Defaults to NULL allowed,
best to define any NOT NULL conditions on the underlying columns. CHECK
constraints can be defined.

\begin{minted}{postgresql}
/* Number to hold 1-10 user rating */
CREATE DOMAIN rating AS integer CHECK ( VALUE >= 1 AND VALUE <=10 );

/* Just use the domain as type when creating table */
CREATE TABLE restaurants (
    id bigserial primary key,
    /* creating two columns using our domain: */
    visitor_rating rating not null,
    reviewer_rating rating, 
    /* other columns */
);
\end{minted}



\section{Foreign keys}\label{foreign-keys}

\href{https://www.postgresql.org/docs/13/ddl-constraints.html\#DDL-CONSTRAINTS-FK}{Foreign
keys} require that values in a column (or a group of columns) must match
the values appearing in some row of another table. This maintains the
referential integrity.

\begin{minted}{postgresql}
/* each employee must be in a valid department */

CREATE TABLE department (
id bigserial primary key,
name text not null unique
/* other fields as required */
);

CREATE TABLE employee (
id bigserial primary key,
surname text not null,
firstname text not null,
department bigint not null REFERENCES department
/* other fields as required */
)
\end{minted}

\subsection{DELETE / UPDATE behaviour}\label{delete-update-behaviour}

Possible behaviours: NO ACTION, CASCADE, SET NULL.

\begin{minted}{postgresql}
CREATE TABLE product ( 
/* other fields */

department bigint references departments, 
/* NO ACTION is the default, prohibits conflicting delete */

supplier bigint references suppliers ON DELETE CASCADE,
/* DELETE in suppliers deletes linked products */

policy bigint references policies ON DELETE SET NULL,
/* SETS product.policy to NULL when row in policies deleted  */

);
\end{minted}

\section{JOIN}\label{join}

The JOIN operation permits queries across more than one table. See both
the
\href{https://www.postgresql.org/docs/current/tutorial-join.html}{JOIN
tutorial} and the
\href{https://www.postgresql.org/docs/13/queries-table-expressions.html}{Table
expressions} section from Postgres manual for full details.

Assume R1 to be a row of Table T1. Similarly R2 for T2. Normally should
explicitly specify columns required and use table prefix to avoid
ambiguity.

\subsection{INNER JOIN}\label{inner-join}

For each row R1 of T1, the joined table has a row for each row in T2
that satisfies the join condition with R1.

\subsection{LEFT JOIN}\label{left-join}

Same as INNER JOIN except that output also includes any row in T1 that
does not match one or more rows in T2. Null values are substituted for
T2 in the output row.

\subsection{RIGHT JOIN}\label{right-join}

Similar to LEFT JOIN. Same as INNER JOIN, except any row in T2 that does
not match \(\ge 1\) rows in T1 will be output. Null values are
subtituted for T1 columns in the otuput row.

\subsection{FULL JOIN}\label{full-join}

Similar to combination of LEFT and RIGHT JOIN. INNER JOIN performed.
Then rows in T1 without corresponding T2 output with nulls for T2. Same
again, rows in T2 without corresponding T1 rows output with nulls for
T1.


\section{Views}\label{sec:views}

\href{https://www.postgresql.org/docs/13/tutorial-views.html}{Views} are
defined by @connolly:2015:database as:

\begin{quote}
The dynamic result of one or more relational operations operating on the
base relations to produce another relation. A view is a virtual relation
that does not necessarily exist in the database but can be produced upon
request by a particular user, at the time of the request.
\end{quote}

\begin{minted}{postgresql}
/* Creation syntax: */
CREATE VIEW my_view AS
SELECT ... ;
/* select statement can be any valid select */

/* VIEW can be selected like any other table */
SELECT * FROM my_view ; 
\end{minted}



\end{document}

