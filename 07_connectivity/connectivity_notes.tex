\documentclass{pgnotes}

\title{Connectivity}

\begin{document}

\maketitle

\section{Database connectivity}
\label{sec:database-connectivity}

As well as text-based \texttt{psql} client, any other client that
implements the postgresql client protocol can connect and use DB server.

Similar access patterns for other RDBMS.

We often will encounter situations where we wish to write a program in a standard language like Python, Java etc that needs to connect to a database.
This is normally done by using a client library, for PostgreSQL this is \texttt{libpq}.

\section{Python on shared server}
\label{sec:python-on-shared-server}

Python can be started interactively at the shell prompt by typing
\texttt{python3}.

Note that \texttt{python} will start the older Python~2 version.
Check the startup banner if you are not sure which Python version you are running.

For writing programs/scripts you just need a text editor.
A number are available: \texttt{nano} is fine, I recommend \texttt{emacs}, others use/like \texttt{vim}.

These can then be executed by exiting the text editor and typing
\begin{minted}{bash}
python3 myscript.py
\end{minted}

Python files can also be executedd directly if they have the header line
\begin{minted}{python}
#!/usr/bin/env python3
\end{minted}
at the top by typing
\begin{minted}{bash}
./myfile.py
\end{minted}
For this to work, they must be made executable first:
\begin{verbatim}
chmod +x ./myfile.py
\end{verbatim}

\section{Psycopg2 connector}
\label{sec:psycopg2-connector}

Python provides a number of ways to access databases, we will use the
\href{https://www.psycopg.org}{Psycopg2} library.

\subsection{Sample program}

\inputminted{python}{connectivity_example.py}

\subsection{Executing statements}
\label{sec:executing-statements}

Statement execution should always use query parameters, either positional or named.
Placeholders in the query are substituted by values passed in, which are often themselves python variables.

\begin{minted}{python}
description = "mow the lawn"
priority = 5

# execute a statement using position parameters
query = "INSERT INTO tasks ( description, priority ) VALUES (%s,%s)"
parameters = ( description, priority,) 
cur.execute(query, parameters)

# alternative using named parameters
query = "INSERT INTO tasks (description, priority) VALUES (%(description)s, %(priority))"
parameters = {'description':description, 'priority':priority}
cur.execute(query, parameters )

# NEVER use string concatenation!
\end{minted}


\section{Lab exercise}

\begin{enumerate}

\item Ensure that you are in a \texttt{tmux} session.
  
\item In one tmux window, use \texttt{psql} to connect to the \texttt{france} database.

\item Write a python program that connects to the \texttt{france} database.
  It should print out the number of towns, regions and departments.
  The output should be: 
\begin{verbatim}
towns: 36684
regions: 26
departments: 100
\end{verbatim}

\item In \texttt{psql}, construct an SQL query to show the following informatin for a particular region.
  Example here is region 26.
\begin{verbatim}
 department_code | department_name | n_towns 
-----------------+-----------------+---------
 21              | Côte-d'Or       |     707
 58              | Nièvre          |     312
 71              | Saône-et-Loire  |     573
 89              | Yonne           |     454
(4 rows)
\end{verbatim}

\item Write a python program that displays all regions.
  It requests the code of a region of a region.
  It should then list out the departments within that region (name and code) and the number of towns in each department.
\begin{verbatim}
number of regions: 26
Alsace [42]
Aquitaine [72]
Auvergne [83]
Basse-Normandie [25]
Bourgogne [26]
Bretagne [53]
Centre [24]
Champagne-Ardenne [21]
... (omitted from handout)
Enter region code: 26
number of departments: 4
21. Cote-d'Or [707]
58. Nievre [312]
71. Saone-et-Loire [573]
89. Yonne [454]
\end{verbatim}

\end{enumerate}
  

\section{PANDAS integration}
\label{sec:pandas-integration}

You are familiar with the use of
\href{https://pandas.pydata.org/}{Pandas} for statistical tasks. PANDAS
can be combined with relational databases to read the output of queries.

\subsection{Dataframe from query}\label{sec:dataframe-from-query}

\inputminted{python}{dataframe_from_query.py}

\section{Exercise}\label{exercise}

\begin{enumerate}
\item
  Construct a task manager (tasks, projects) {[} datatypes, null/not
  null, keys, foreign keys, domains + enumerated types {]} for yourself.
  (Alternatives: expense tracker, sports score, track exercise -
  anything with *2* or more tables.) Set up the tables using psql.
\item
  If you haven't already, set a password using psql's
  \texttt{\textbackslash{}password}.
\item
  Write Python program (in nano, emacs, vim) to:

  \begin{enumerate}
  \def\labelenumii{\arabic{enumii}.}
  \item
    Allow user to create either a new task or a new project
  \item
    If a new project, get the name \& then exit.
  \item
    If a new task, display list of projects.
  \item
    Ask user to choose which project new project task belongs to.
  \item
    Ask user name of new task.
  \item
    Then ask to save or not. If yes, commit, if not, rollback.
  \end{enumerate}
\item
  Write another Python program to display a query using Pandas (for
  example a VIEW showing tasks and their projects JOINed).
\end{enumerate}

\end{document}


