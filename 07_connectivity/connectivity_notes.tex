\documentclass[slides]{pgnotes}

\title{Connectivity}

\begin{document}

\maketitle

\section{Database connectivity}
\label{sec:database-connectivity}

As well as text-based \texttt{psql} client, any other client that
implements the postgresql client protocol can connect and use DB server.

Similar access patterns for other RDBMS.

We often will encounter situations where we wish to write a program in a standard language like Python, Java etc that needs to connect to a database.
This is normally done by using a client library, for PostgreSQL this is \texttt{libpq}.

\section{Python on shared server}
\label{sec:python-on-shared-server}

Python can be started interactively at the shell prompt by typing
\texttt{python3}.

\begin{redbox}{Use the correct Python}
  \begin{itemize}
  \item Note that \texttt{python} will start the older Python~2 version.
  \item Check the startup banner if you are not sure which Python version you are running.
  \end{itemize}
\end{redbox}

\subsection{Text editor}

For writing programs/scripts you need a \textbf{text editor}.

Learning how to use text-user-interface editors is a good skill to pick up as a data analyst.

\begin{greenbox}{Available editors}
  \begin{itemize}
  \item \texttt{nano} is fine
  \item I recommend \texttt{emacs} (which has a Windows GUI version as well)
  \item Others use/like \texttt{vim}.
  \end{itemize}
\end{greenbox}

\subsection{Script execution}

Once saved, python programs can then be executed by exiting the text editor and typing:

\begin{minted}{bash}
python3 myscript.py
\end{minted}

Python files can also be executedd directly if they have the header line
\begin{minted}{python}
#!/usr/bin/env python3
\end{minted}
at the top by typing
\begin{minted}{bash}
./myfile.py
\end{minted}
For this to work, they must be made executable first:
\begin{verbatim}
chmod +x ./myfile.py
\end{verbatim}

\section{Psycopg2 connector}
\label{sec:psycopg2-connector}

Python provides a number of ways to access databases, we will use the
\href{https://www.psycopg.org}{Psycopg2} library.

\subsection{Sample program}

\inputminted{python}{connectivity_example.py}

\subsection{Executing statements}
\label{sec:executing-statements}

Statement execution should always use query parameters, either positional or named.
Placeholders in the query are substituted by values passed in, which are often themselves python variables.

\begin{minted}{python}
description = "mow the lawn"
priority = 5

# execute a statement using position parameters
query = "INSERT INTO tasks ( description, priority ) VALUES (%s,%s)"
parameters = ( description, priority,) 
cur.execute(query, parameters)

# alternative using named parameters
query = "INSERT INTO tasks (description, priority) VALUES (%(description)s, %(priority))"
parameters = {'description':description, 'priority':priority}
cur.execute(query, parameters )

# NEVER use string concatenation!
\end{minted}

  

\section{PANDAS integration}
\label{sec:pandas-integration}

You are familiar with the use of
\href{https://pandas.pydata.org/}{Pandas} for statistical tasks. PANDAS
can be combined with relational databases to read the output of queries.

\subsection{Dataframe from query}\label{sec:dataframe-from-query}

\inputminted{python}{dataframe_from_query.py}


\end{document}


