\documentclass[slides]{pgnotes}

\title{Secure SHell (SSH)}

\begin{document}

\maketitle

\tableofcontents

\section{Interfaces}

The way we interact with computers has changed over time.
There are now three main interface types.

\begin{bluebox}{Graphical}
\begin{description}
  
\item[Multi-touch:] via a touchscreen usually on portable device 
  \begin{itemize}
  \item Many people use a phone / tablet as their primary / only personal computer.
  \end{itemize}

\item[Desktop GUI:] familiar to most users
\end{description}
\end{bluebox}
 
\begin{bluebox}{Text-based}
Often seen in older systems and by more advanced users:

  \begin{description}
  \item[Screen-based UI] where applications resemble Desktop GUIs
  \item[Command-line] where programs are run in a command-shell
  \end{description}

  Many text-based UI users encounter both styles depending on context.
\end{bluebox}

\section{Shell}

The \textbf{shell} is the program we normally interact with in a command-line interface.

\begin{bluebox}{Shells on Windows}
  \begin{itemize}
  \item PowerShell (current)
  \item Command.com (older)
  \end{itemize}
\end{bluebox}

\begin{bluebox}{Shells on UNIX}
  \begin{itemize}
  \item Most common: \texttt{bash}
  \item New and feature enhanced: \texttt{zsh}
  \item Also: Korn Shell, C shell
  \end{itemize}
\end{bluebox}

\subsection{Key concepts}

\begin{description}
\item[Prompt] shows when shell is waiting for input.
\item[Current working directory] where commands will read and write files relative to.
\item[Path:] list of folders searched for matching command name
\end{description}

\subsection{Navigation}

In the command-line environment we navigate the exact same set of folders as we see in the File Explorer / Finder.
Some hints on navigation (applies to PowerShell and Bash):

\begin{minted}{bash}
# print out the folder you're in (i.e. the working directory)
pwd

# list out the contents of the folder you're in
ls
dir   # on windows
ls -l # detail, linux/mac only

# change to a sub-directory of where you are now
cd movies

# directly change to a sub-sub-directory
cd movies/horror

# change to the parent directory
cd ..

# change to your home directory
cd ~

# change to a known sub-dir of your home directory
cd ~/Desktop

# change to a sub-dir of the parent dir
cd ../music
\end{minted}

\textbf{It is essential in command-line environments that you are comfortable navigating around the filesystem.}

\subsection{Features}

\begin{description}
\item[History:] list of previous commands recalled (usually the up arrow key).
\item[Redirection] using \textit{operators}
  \begin{enumerate}
  \item Standard input to a file.
  \item Standard input from a file.
  \item Piping the standard output of one command to the standard input of another.
  \end{enumerate}
\item[Scripting] a sequence of commands to be performed.
\item[Variables] to capture and recall information.
\item[Control constructs] including conditionals, loops, possibly exceptions.
\end{description}

\section{Terminal}

The shell itself is normally accessed by means of a terminal.
This is the program we visually see like the PowerShell Application or XTerm in Linux that encapsulates the terminal program with the GUI environment.
Examples of terminals:

\begin{description}
\item[GUI terminals ] like Windows Terminal, XTerm, Terminal.app
\item[Framebuffer console] on Windows when the GUI is not running.
\item[Serial console] over a serial port (often seen on embedded devices).
\item[Remote network terminals] using telnet or more usually SSH.
\end{description}

\section{Secure Shell}

SSH is a way to for one computer to connect to another's command-line interface in a secure fashion.

SSH clients are included in most common operating systems.
Client apps available for mobile. 

An SSH client connects to an SSH server.
The SSH server normally makes the command-line interface of the OS available (e.g.~bash, powershell):
\begin{itemize}
\item All modern UNIX/Linux operating systems come with SSH servers as standard.
\item Windows 10 onwards and Windows Server now have SSH servers included but need some configuration to get working.
\end{itemize}

SSH is relatively easy to get started with --- the complexity often comes
later when features like key-based authentication, multi-factor
authentication, port forwarding and other extras are employed.

\autoimage{ssh}{SSH}{ssh}

\section{SSH client}
\label{ssh-client}

Most operating systems use the OpenSSH client, named \texttt{ssh}, that is available on the command-line.

\subsection{Connecting}

To connect to a remote machine, we need to know its name or IP and the username to connect as:

\begin{minted}{bash}
# connect via IP
ssh peadar@192.168.0.1

# connect via name
ssh peadar@compute-server.dkit.ie

# connect using same username as on client
# just leave off the username before @ symbol
ssh 192.168.0.1
ssh compute-server.dkit.ie
\end{minted}

\subsection{Host verification}

The first time you connect to a host you'll get a warning:

\begin{verbatim}
The authenticity of host '54.78.220.233 (54.78.220.233)' can't be established.
ECDSA key fingerprint is SHA256:8omkD5RLibZNgJJ/B7MAnL7IbEcrmCmIWFdQXbjJf60.
Are you sure you want to continue connecting (yes/no)?
\end{verbatim}

Just type \texttt{yes} here:
\begin{itemize}
\item Your local SSH client is just confirming it hasn't seen this machine before.
\item If a different key fingerprint shows for the same IP you'll get a warning, which means a host has been changed for another.
\end{itemize}


\subsection{Authentication}

SSH supports a number of different authentication schemes:
\begin{itemize}
\item A server may permit or require multiple authentication methods.
\item Simplest is \textbf{username / password}.
\item SSH often used with \textbf{Key Pairs} (later on).
\item Other authentication methods:
  \begin{description}
  \item[Authenticator apps] like Google Authenticator.
  \item[Kerberos] where Windows AD can ``pass through'' authentication from client.
  \end{description}
\end{itemize}

\subsection{Usage}

If you see something like the following (on Linux) then you're connected:

\begin{verbatim}
       __|  __|_  )
       _|  (     /   Amazon Linux 2 AMI
      ___|\___|___|

https://aws.amazon.com/amazon-linux-2/
2 package(s) needed for security, out of 13 available
Run "sudo yum update" to apply all updates.
[student@ip-10-0-1-80 ~]$
\end{verbatim}

What will actually appear will depend on what type of host you are connecting to.


\section{File transfer}

\subsection{Connecting}

To connect to a remote machine to transfer files (instead of using its command line) we just replace \texttt{ssh} with \texttt{sftp} in the command:

\begin{minted}{bash}
sftp username@ip-address-or-hostname-here
\end{minted}

Login is the same as with normal SSH.

When connected you'll see the \texttt{sftp>} prompt.


\subsection{Directories}

When connecting over SFTP you are dealing with \textbf{2} working directories:
\begin{description}
\item[Remote working directory ] on the remote server
  \begin{itemize}
  \item Use \texttt{pwd} to show.
  \item Use \texttt{cd} to navigate around.
  \end{itemize}
\item[Local working directory] on the client computer you're using
  \begin{itemize}
  \item Use \texttt{lpwd} to show.
  \item Use \texttt{lcd} to navigate around.
  \end{itemize}
\end{description}
It is more confusing when the remote and local computers are using the same operating system, since the paths won't naturally differ.


\end{document}



