\documentclass[slides]{pgnotes}

\title{Single Table}

\begin{document}

\maketitle

\tableofcontents

\section{Introduction}
\label{sec:introduction}

This week's class covers the setup and usage of a basic single-table database.

As a data analyst:

\begin{itemize}
\item You will often encounter situations where you will run queries on databases that already exist.
\item Other times you will be tasked with setting up a database to conduct one-off or ongoing analyses.
\end{itemize}
  
We will therefore look at how data needs to be modelled for a \textbf{single table}, focusing on:
\begin{enumerate}
\item Column names
\item Data types
\item Constraints to improve data integrity.
\end{enumerate}




\subsection{Prerequisities}
\label{sec:prerequisities}

Before starting make sure that you:

\begin{itemize}
\item
  created your SSH public/private key-pair.
\item
  can login over SSH to the shared server.
\end{itemize}

\begin{redbox}{If not}
  Talk to me immediately!
\end{redbox}


\section{Single-table databases}

Recall that:

\begin{itemize}
\item A PostgreSQL \textit{cluster} provides:
\item One or more \textbf{databases} that each have
\item (One or more \textbf{schemas} that each hold)
\item One or more \textbf{tables}
\end{itemize}

where each table:

\begin{itemize}
\item has separate \textbf{columns} holding individual attribute values
\item stores data each record in \textbf{rows}
\end{itemize}


\subsection{Data definition language}

Key concepts will be easily learned through repeated practice and
reference to

\begin{description}
\item[CREATE table]
  to define table name, columns (type, constraints)
\item[ALTER table]
  for incremental changes.
\item[DROP table]
  without warning!
\end{description}

Sometimes it's easiest to specify information up front in the \texttt{CREATE} statement.

Often it's better to start with a workable definition and amend using\texttt{ALTER}.

\subsection{Sample problem}

We will mainly do this by example.

\begin{greenbox}{Example}
  Mary is a local crafter who makes her own pottery.
  She runs a small shop where she sells her own wares.

  At the moment she writes down all sales in a book:
  \begin{itemize}
  \item Date and sometimes time
  \item Description of what sold
  \item Quantity
  \item Price
  \item Total
  \end{itemize}

  Having done a database class, she realises that she could easily put together a table to replace the book.
\end{greenbox}

\subsection{Table names}

Following a consistent and workable pattern with table names will help a lot:

\begin{enumerate}
\item names should be \textbf{descriptive}
  \begin{itemize}
  \item decide on a sensible structure on pluralisation
  \end{itemize}
\item try to keep names reasonablly \textbf{short}
  \begin{itemize}
  \item You will be typing them a lot!
  \end{itemize}
\item do not use spaces, use the underscore instead (\_)
  \begin{itemize}
  \item spaces in most text-based systems are a receipe for disaster!
  \end{itemize}
\item table names are \textbf{case insensitive}
\end{enumerate}

Table names \textit{can} be changed afterwards.

Our table will be called \textbf{sales}.


\subsection{Column names}

Similar set of rules for columns as for tables:

Again following a consistent and workable pattern with column names will help a lot:

\begin{enumerate}
\item names should be \textbf{descriptive}
\item try to keep names reasonablly \textbf{short}
\item do not use spaces, use the underscore instead (\_)
\end{enumerate}

Column names \textit{can} be changed afterwards.

\newpage

\begin{greenbox}{Example}

  In our example, we'll choose the following column names:

  \begin{enumerate}
  \item \texttt{timestamp}
  \item \texttt{description}
  \item \texttt{quantity}
  \item \texttt{total\_price}
  \end{enumerate}

\end{greenbox}

\section{Datatypes and constraints} 

\textbf{Choosing the correct datatypes is vitally important for making a usable database!}

\begin{description}
\item[TEXT] for text (only!)
  \begin{itemize}
  \item Other database systems use CHAR, VARCHAR etc. Avoid!
  \end{itemize}
\item Numeric data largely falls into the categories of: integer, fixed precision and arbitary precision data.
  \begin{description}
  \item[INT] for integers
    \begin{itemize}
    \item Also have \texttt{SMALLINT, BIGINT} etc for range
    \end{itemize}
  \item[NUMERIC] for numeric data incl. decimals and currency
    \begin{itemize}
    \item Easy to make mistakes choosing float instead!
    \end{itemize}
  \end{description}
\end{description}

\subsection{Good practices}

\begin{enumerate}
  
\item Do not store numbers as text.

\item Do not store boolean true / false ( or any synonyms ) as text.

\item Do not sure enumerated types as text. Either use an ENUM and/or a foreign-keyed table.

\item Anything where you need absolute precision after the decimal point must be NUMERIC, not FLOAT. 
  
\item Do not be tempted to store a single logical date/time as separate date and time columns.
  Always use a single timestamp for this. 
  
\end{enumerate}


\subsection{NOT NULL}\label{not-null}

The NOT NULL constraint prevents a particular column value being null.
Writes will fail if set to NULL or if unspecified unless default column value specified.

\begin{verbatim}
-- During table creation:
CREATE TABLE tasks (
   description text not null,
   /* other columns */
);
\end{verbatim}

\newpage
\subsubsection{Changing NOT NULL status}

Whether NULL should be allowed depends on problem of interest.
By default NULL is allowed.
Must specify NOT NULL if not.

Can modify the column's NULLable status afterwards:
\begin{verbatim}
-- make a column NOT NULL
ALTER TABLE tasks ALTER COLUMN description SET NOT NULL

-- remove the NOT NULL constraint from a column
ALTER TABLE tasks ALTER COLUMN description DROP NOT NULL
\end{verbatim}

\begin{greenbox}{Example}
  In our example we will set all columns to not null, as none are optional.
\end{greenbox}


\subsection{UNIQUE}\label{unique}

The UNIQUE constraint prohibits two rows from having an equal set of
attribute values for the columns specified:

\begin{itemize}
\item Use UNIQUE after column definition if column should be unique
\item Use UNIQUE(col1, col2) if two or more columns should be unique after the column definitions
\end{itemize}

\newpage

\begin{verbatim}
-- During table creation:
CREATE TABLE tasks (
   description text unique,
   /* other columns as necessary */
   project_code text,
   subproject_code text,
   unique(project,subproject)
); 
\end{verbatim}

\begin{greenbox}{Example}
  In our example we've no UNIQUE columns (yet!)
\end{greenbox}

\newpage
\subsubsection{Modifying UNIQUE afterwards}

Just as with NOT NULL we can modify UNIQUE later on:

\begin{verbatim}
-- Afterwards
ALTER TABLE tasks ADD UNIQUE (project, subproject);
\end{verbatim}


\subsection{Default values}
\label{sec:default-values}

Default values are automatically inserted when unspecified in INSERT
statement.
Often used in conjunction with NOT NULL.
Can be static value or based on functions (see later).

\begin{verbatim}
-- when creating a table
CREATE TABLE sales (
    /* other columns  */
    quantity smallint not null default 1;
    timestamp timestamp not null default now(); 
);
\end{verbatim}


\newpage
\subsubsection{Altering DEFAULT}

\begin{verbatim}
-- afterwards
ALTER TABLE tasks ALTER COLUMN description SET default 'xyz'
ALTER TABLE tasks ALTER COLUMN description DROP default
\end{verbatim}


\section{Primary key}
\label{sec:primary-key}

\textbf{Primary key} value is a \textbf{unique} AND \textbf{not null}.

Unambiguously identifies each row.

Every table should have 1 primary key.
Cannot have more than 1. 

Most DBMS do not enforce this, but you should consider it mandatory.
Use PRIMARY KEY if column is the primary key. 

\begin{greenbox}{Example}
  In our example, we \textit{could} use the timestamp as the primary key.

  But it would be much better to make an autoincrementing integer \texttt{id} column.
\end{greenbox}

\newpage

\begin{verbatim}
-- specifying primary key
CREATE TABLE accounts ( 
    account_number bigint primary key
    /* other columns as necessary */
);

-- primary key covering two columns
CREATE TABLE accounts (
    branch bigint, -- not null automatic
    account_number bigint,  -- not null automatic
    /* other columns */
    primary key (branch, account_number)
    /* other constraints etc. */
)
\end{verbatim}

\subsection{Auto-increment column}
\label{sec:auto-increment-column}

We can use the column types SERIAL or BIGSERIAL for auto-incrementing columns.
Generally prefer SERIAL or BIGSERIAL.

Shorthand for default value based on \textit{sequence generator} (later on).

\begin{verbatim}
CREATE TABLE sales ( 
    id BIGSERIAL PRIMARY KEY,
    /* other columns & constraints */
);
\end{verbatim}


\section{Transaction control}
\label{sec:transaction-control}

\href{https://www.postgresql.org/docs/15/tutorial-transactions.html}{PostgreSQL
transaction control} can be used in simplest form to give basic ``undo''
capability:

\begin{verbatim}
-- start a new transaction
BEGIN;

/* execute SQL statements then either: */

COMMIT; /* save the changes */
-- or
ROLLBACK; /* undo the changes */
\end{verbatim}

\subsection{Error handling}
\label{sec:error-handling}

If we don't BEGIN a transaction we implicitly BEGIN before the statement
and COMMIT after it. Errors (e.g.~syntax) will abort the transaction,
requiring a ROLLBACK before any more statements will be accepted.

\subsection{Savepoints}\label{savepoints}

\begin{verbatim}
BEGIN;

/* sql statement block 1 */

-- define a savepoint
SAVEPOINT sp1;

/* sql statement block 2 */

-- rollback to the savepoint undoes block 2; 
ROLLBACK TO sp1;

-- then either:
COMMIT; /* or ROLLBACK */
\end{verbatim}

\section{DML operations}

The main \textbf{Data Manipulation Language} operations you'll use are grouped into:

\begin{description}
\item[INSERT]
  rows populating specified columns with data given.
\item[UPDATE]
  table, optionally select rows with \texttt{WHERE}.
\item[DELETE]
  rows from a table, optionally select rows with \texttt{WHERE}.
\end{description}

We'll mainly cover these by example!

\end{document}

